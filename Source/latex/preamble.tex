%Dokumentklasse
\documentclass[a4paper,12pt,makeidx,twoside,openright,numbers=noenddot]{scrreprt}
\usepackage[left=3cm, right=2cm, bottom=2.5cm, top=3cm]{geometry}		% Formatierung der Seitenränder
%\usepackage[onehalfspacing]{setspace}							        % verwendet für größeren Zeilenabstand

% ============= Packages =============

% Dokumentinformationen


\usepackage[
	pdftitle={},
	pdfsubject={},
	pdfauthor={},
	pdfkeywords={},
	hidelinks
]{hyperref}



\usepackage[utf8]{inputenc}
\usepackage[ngerman]{babel}
\usepackage[T1]{fontenc}
\usepackage{units}
\usepackage{pdfpages}
\usepackage{listings}
\usepackage{svg}					% Zur Einbindung von scalable vector graphics
\usepackage{subcaption}				% Zum Einbinden von Untergrafiken
\usepackage[gen]{eurosym}			% Zur Verwendung des Eurozeichens
\usepackage{amssymb}
\usepackage{graphicx}
\graphicspath{{img/}}				
\usepackage{fancyhdr}				% Style Package für's Seitenlayout
\usepackage{color}					% anpassen von Farben
\usepackage{microtype} 				% schönerer Blocksatz!
\usepackage{calc}  					
\usepackage{enumitem}				% zb für align der description
\usepackage[font=small,labelfont=bf]{caption}
\usepackage[printonlyused]{acronym}	% für Abkürzungen
\usepackage{multicol}
\usepackage{booktabs}
\usepackage{textcomp}
\usepackage{listings}				% Einbindung von Code in LateX
\usepackage{setspace}
\usepackage{threeparttable} 		% für Fußnoten innerhalb einer Tabelle
\usepackage{tikz}
\usepackage{enumitem}
\usepackage{multirow}
\usepackage{comment}
%\usepackage{minipage} 
%\usepackage{floatrow} 
% CODE PLUGIN!!
\usepackage{minted}

% verschiedene Schriftarten
%\usepackage{times} 				% times font
%\usepackage{palatino}			 	% Palatino font
\usepackage{lmodern}				% Lmodern sans und serif
%\usepackage{libertine}				% Linux Libertine und Biolinum

%\usepackage{fontspec}				% Nutzen in Kombination mit LuaLaTeX, um Systemschriften einzubinden
%\setmainfont{Georgia}				% beispielsweise Georgia, aber auch jede andere Schrift, die auf dem PC vorhanden ist


% zusätzliche Schriftzeichen der American Mathematical Society
\usepackage{amsfonts}
\usepackage{amsmath}

\usetikzlibrary{%
  arrows,%
  calc,
  shapes,
  arrows,
  shapes.misc,% wg. rounded rectangle
  shapes.arrows,%
  chains,%
  matrix,%
  positioning,% wg. " of "
  scopes,%
  decorations.pathmorphing,% /pgf/decoration/random steps | erste Graphik
  shadows%
}
\usepackage{amsmath}
\usepackage{relsize}

\usepackage[numbers, comma]{natbib}		% Einstellung des Zitierstils
%\bibliographystyle{myabbrvnat}			% Angepasster Stil für deutsche Sprache



\setcounter{secnumdepth}{3}				% Nummerierungsebene anpassen -> 3 = subsubsection werden nummeriert
\setcounter{tocdepth}{2}   				% gliederungsebenen im Inhaltsverzeichnis -> erstmal nur zur übersicht was nicht vergessen werden darf

\definecolor{deepblue}{rgb}{0,0,0.5}
\definecolor{deepred}{rgb}{0.6,0,0}
\definecolor{deepgreen}{rgb}{0,0.5,0}


% ============= Kopf- und Fußzeile =============
\pagestyle{fancy}

%% Formatierung der Kopf- und Fußzeile
\fancyhead{}
\fancyhead[RO,LE]{\thepage}
\fancyhead[RE]{\leftmark}
\fancyhead[LO]{\rightmark}
%%
\fancyfoot{}

\renewcommand{\headrulewidth}{0.4pt}		% Bei zweiseitigem Dokument ausschließlich Linie in Kopfzeile
\renewcommand{\chaptermark}[1]{\markboth{\thechapter\ #1}{}}
\renewcommand{\sectionmark}[1]{\markright{\thesection\ #1}}

% ============= Package Einstellungen & Sonstiges ============= 
% Besondere Trennungen
\hyphenation{Um-ge-bungs-tem-pe-ra-tur Um-ge-bungs-tem-pe-ra-tur-en Rauch-gas-tem-pe-ra-tur Aus-tritts-tem-pe-ra-tur}

% Einstellung wie Code innerhalb der Arbeit gesetzt werden soll:
\lstdefinestyle{Style}{
	columns=flexible,
	basicstyle=\ttfamily}
\lstset{ 
	backgroundcolor=\color{white},   % choose the background color; you must add \usepackage{color} or \usepackage{xcolor}; should come as last argument
	basicstyle=\footnotesize,        % the size of the fonts that are used for the code
	breakatwhitespace=false,         % sets if automatic breaks should only happen at whitespace
	breaklines=true,                 % sets automatic line breaking
	captionpos=none,                 % sets the caption-position to bottom
	commentstyle=\color{deepblue},   % comment style
	deletekeywords={...},            % if you want to delete keywords from the given language
	escapeinside={\%*}{*)},          % if you want to add LaTeX within your code
	extendedchars=true,              % lets you use non-ASCII characters; for 8-bits encodings only, does not work with UTF-8
	firstnumber=1,               	 % start line enumeration with line 1000
	frame=single,	                 % adds a frame around the code
	keepspaces=true,                 % keeps spaces in text, useful for keeping indentation of code (possibly needs columns=flexible)
	keywordstyle=\color{blue},       % keyword style
	language=Python,                 % the language of the code
	morekeywords={*,...},            % if you want to add more keywords to the set
	numbers=left,                    % where to put the line-numbers; possible values are (none, left, right)
	numbersep=5pt,                   % how far the line-numbers are from the code
	emphstyle=\color{deepred},
	%numberstyle=\tiny\color{mygray}, % the style that is used for the line-numbers
	rulecolor=\color{black},         % if not set, the frame-color may be changed on line-breaks within not-black text (e.g. comments (green here))
	showspaces=false,                % show spaces everywhere adding particular underscores; it overrides 'showstringspaces'
	showstringspaces=false,          % underline spaces within strings only
	showtabs=false,                  % show tabs within strings adding particular underscores
	stepnumber=2,                    % the step between two line-numbers. If it's 1, each line will be numbered
	stringstyle=\color{deepgreen},     % string literal style
	tabsize=2,	                   	 % sets default tabsize to 2 spaces
	title=\lstname                   % show the filename of files included with \lstinputlisting; also try caption instead of title
}
% nicht einrücken nach Absatz
\setlength{\parindent}{0pt}
\usepackage{parskip}		 			% verhindert einrücken und setzt einen kleinen Absatz

\renewcommand{\arraystretch}{1.2}		% Abstand innerhalb der Tabellen einstellen