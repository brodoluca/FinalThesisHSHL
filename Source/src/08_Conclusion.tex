\chapter{Overview and Future Work}

Sugar beet plantations, due to their poor performances against other competitors like weed, will benefit from the advancements brought by the introduction of smart farming techniques. Most of the solutions in this field proposed methodologies for increasing pro- duction and sustainability based on the recognition and localization of sugar beets using Neural Networks. Neural Networks, however, impose obstacles which, despite their great flexibility, make tailoring applications around them challenging.\\
In this paper, we investigated how the analysis of the characteristics of Neural Networks brings to find correlations between them that can be used to predict the performances of the networks in a time-wise reliable way.\\
We started our investigation in chapter \ref{char_nn} by defining which characteristics we were going to use and which could be beneficial in the context of sugar beet recognition. By defining these characteristics, we posed our foundations for the development of the tool which allows for further analysis. To develop this tool, in chapter \ref{ana_models}, we explored benchmarking techniques and the characteristics that we would like to have for a correct test environment. With the help of this tool, in chapter \ref{ana_char},  we studied two use cases to demonstrate how we can effectively find these correlations and these characteristics, mentioning how they could be used in future applications. \\
Even though in the two analyses we achieved promising and insightful results, this paper leaves many interrogatives open which need to be explored in future works. \\
First of all, in chapter \ref{char_nn}, we only investigated some of the characteristics which neural networks possess and can be correlated with each other. For example, we did not focus on parameters which influence the performances of the networks, like for e.g. learning rate, batch size or throughput. It is left for further studies to identify other characteristics that can be worth studying to find more correlations.\\
In chapter \ref{ana_models}, only techniques related to measuring execution time have been described. However, benchmarks are also used to measure other metrics, such as energy consumption or memory use. It is left for future work to find better techniques to precisely measure and monitor them, as they are also of the highest importance especially for devices with low on-board resources. Furthermore, in this paper we mainly focused on Linux, therefore purposely neglecting other operating systems, only mentioning some techniques to run benchmarks on mobile devices running Android. The study of measuring techniques for other systems is also left for future work. \\
In chapter \ref{ana_char}, we run the benchmarking tool only on two examples. Even though we obtained promising results, we left some interrogatives behind. Firstly, we only ran the tool with models from three different CNN architectures. Secondly, we did not investigate how ad-hoc-created models will behave under the conditions we defined. Thirdly, the experiments we made have been run without any augmentation or optimization on both the models and datasets. Regarding the datasets used, we only touched the surface regarding image processing and data augmentation. As a matter of fact, we only considered grey-scale version of the pictures, leaving further analysis for future work. It would be interesting to analyse how the augmentation and modifications of these pictures, like for e.g. blurring, cropping and rotation, will influence the behaviour of the models. Furthermore, it would also be insightful to develop a tool to analyse other characteristics of the images (colour concentration, compression algorithm, etc.) to find correlations between those and inference time. Regarding the models have been trained with the standard batch size, learning rate and were not pre-trained. In addition, they have been trained with full precision. It is left for further investigations to prove if further correlations can be found using different models and different optimizations.\\
Finally, we only investigated if, and which, correlations can be found studying the performance and the behaviour of neural networks. However, this serves only as a starting point for further applications and we did not investigate if these correlations can be used to optimise real world applications. Further research would be needed to prove it. \\
The result of our journey poses the first milestone towards the use of characteristics and correlations to make sugar beet recognition more efficient, simpler and more reliable with the hope that it could help to improve production and sustainability of the sugar beet plantations around the world. 

