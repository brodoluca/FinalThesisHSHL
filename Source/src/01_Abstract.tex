
\begin{abstract}
Sugar beets plantations are perfect candidates for application of smart farming methodologies. While bringing innumerable benefits, most of the already proposed solutions are based on the recognition of the plants to avoid decreasing the yield and to increase sustainability. As the recognition task is usually tackled using Convolutional Neural Networks (CNNs), the unpredictability of those impose challenges to the engineers. In this paper, we focus on the analysis of the networks to find correlations between their characteristics to be reliable and predictable in time, which can allow engineers to tailor the CNNs for their applications. We will start our investigation from defining and analysing the characteristics we are interested in. This step allows us to find appropriate techniques to create a benchmark tool for analysing neural networks. We will use this tool and the characteristics we define to investigate the training and inference process of eight models using two different dataset.\\
Although these two use cases show promising results, as we are only going to focus on analysing models and finding correlations between some of their characteristics, this paper is only the first milestone towards using this correlation to optimize smart farming applications to improve the production and sustainability of the plantations around the world. 
\end{abstract}